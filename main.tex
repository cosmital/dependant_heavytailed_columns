\documentclass[ECP, preprint]{ejpecp} % replace ECP by EJP if needed.

\usepackage{amsmath}
\usepackage{amstext}
\usepackage{amsfonts}
\usepackage{amssymb}
\usepackage{graphicx}
\usepackage{bbm}
% \usepackage{CJK}
\usepackage{tikz}
\usepackage{pgfplots}
\usepackage[T2A,T1]{fontenc}
\newcommand\restrict[2]{{#1}\raisebox{-.5ex}{$|$}_{#2}}


\usepackage{calc} 
\newcommand{\op}{ \mathop{ \vphantom{\bigoplus} \mathchoice {\vcenter{\hbox{\resizebox{\widthof{$\displaystyle\bigoplus$}}{!}{$\boxplus$}}}} {\vcenter{\hbox{\resizebox{\widthof{$\bigoplus$}}{!}{$\boxplus$}}}} {\vcenter{\hbox{\resizebox{\widthof{$\scriptstyle\oplus$}}{!}{$\boxplus$}}}} {\vcenter{\hbox{\resizebox{\widthof{$\scriptscriptstyle\oplus$}}{!}{$\boxplus$}}}} }\displaylimits } 

\renewcommand{\arraystretch}{1.5}
% \usepackage[
% backend=biber,
% style=alphabetic,
% sorting=ynt
% ]{biblatex}
% \addbibresource{D:/OneDrive - CUHK-Shenzhen/Documents/Research/biblio.bib}

% \bibliography{D:/OneDrive - CUHK-Shenzhen/Documents/Research/biblio}

\newcommand\script[1]{{\fontfamily{pzc}\fontshape{it}\selectfont#1}}
\SHORTTITLE{Operations \& concentration}

\TITLE{Eigen behaviour of Random matrices with \\ Heavy tailed independent columns}
    %\thanks{Current maintainer of class file is
     % \href{https://vtex.lt}{VTeX, Lithuania}. Please send all queries to
     % \href{mailto:latex-support@vtex.lt}{\texttt{latex-support@vtex.lt}}.}} % \thanks is optional. Insert line breaks with \\

%\DEDICATORY{Dedicated to the memory of ...} % Optional

%%%%%%%%%%%%%%%%%%%%%%%%%%%%%%%%%%%%%%%%%%%%%%%%%%%%%%%%%%%%%%%%%%%
%%                                                               %%
%% Authors (please edit and customize):                          %%
%%                                                               %%
%%%%%%%%%%%%%%%%%%%%%%%%%%%%%%%%%%%%%%%%%%%%%%%%%%%%%%%%%%%%%%%%%%%
% \AUTHORS{%
%   Cosme Louart\footnote{ CUHK (Shenzhen), School of data science. \EMAIL{cosmelouart@cuhk.edu.cn}}
%   \and %% remove this line and below if single author
%   }
\AUTHORS{Authors\footnote{School of Data Science, The Chinese University of Hong Kong (Shenzhen), Shenzhen, China}}
%\and%%
%Romain Couillet\footnote{LIG-lab,
%GIPSA-lab. \EMAIL{romain.couillet@gipsa-lab.grenoble-inp.fr}}}
%% Type \and between all consecutive authors (not only before the last author).
%% Note: you may use \BEMAIL to force a line break before e-mail display.

%% Here is a compact example with two authors with same affiliation
%% \AUTHORS{%
%%  Michael~First\footnote{Some University. \EMAIL{mf,js@uni.edu}
%%  \and
%%  John~Second\footnotemark[2]}%AUTHORS
%% Note: The \footnotemark is the footnote number that you wish to reuse. Here
%% it is [2] (we took into account the footnote generated by \thanks in title).

%%%%%%%%%%%%%%%%%%%%%%%%%%%%%%%%%%%%%%%%%%%%%%%%%%%%%%%%%%%%%%%%%%%
%%                                                               %%
%% Please edit and customize the following items:                %%
%%                                                               %%
%%%%%%%%%%%%%%%%%%%%%%%%%%%%%%%%%%%%%%%%%%%%%%%%%%%%%%%%%%%%%%%%%%%

\KEYWORDS{Random matrix; Heavy tailed concentration; Hanson-Wright inequality} % Separate items with ;

\AMSSUBJ{60-08, 60B20, 62J07} % Edit. Separate items with ;
%\AMSSUBJSECONDARY{FIXME:} % Optional, separate items with ;

\SUBMITTED{} % Edit.
%\ACCEPTED{} % Edit.

%%%%%%%%%%%%%%%%%%%%%%%%%%%%%%%%%%%%%%%%%%%%%%%%%%%%%%%%%%%%%%%%%%%
%%                                                               %%
%% Please uncomment and edit if you have an arXiv ID:            %%
%%                                                               %%
%%%%%%%%%%%%%%%%%%%%%%%%%%%%%%%%%%%%%%%%%%%%%%%%%%%%%%%%%%%%%%%%%%%

%\ARXIVID{2102.08020} % Edit.

\ABSTRACT{Abstract}
\begin{document}
helloe try
\section{The semi-metric and Lipschitz mapping}
We introduce the semi-metric $d_s$ on $\mathcal{D}_n(\mathbb{H})=\{D\in \mathcal{D}_n,\forall i \in [n],\Im{D}_i>0\}$:
$$d_s(\Delta,\Delta')=\sup_{1\leq i\leq n}\frac{|\Delta-\Delta'|}{\sqrt{\Im(\Delta)\Im(\Delta')}}$$

The distance $d_s$ is not a metric because it does not satisfy the triangular inequality, see the following counter-example:
$$d_s(4i,i)=\frac{3}{2}>\frac{1}{\sqrt{2}}+\frac{1}{\sqrt{2}}=d_s(4i,2i)+d_s(2i,1i)$$

Indeed, one has the counter-triangular inequality when certain conditions are met:
\begin{lemma}
    Given $x,y,z\in \mathbb{R}$, $x<y<z$ implies that:
    $$d^2_s(a+xi,a+zi)>d^2_s(a+xi,a+yi)+d^2_s(a+yi,a+zi)$$
\end{lemma}
\begin{proof}
    Here we construct the function
    $$g:y\to \frac{(y-x)^2}{xy}+\frac{(z-y)^2}{yz}$$
and we differentiate it twice to get:
$$g'(y)=\frac{y^2-x^2}{xy^2}+\frac{y^2-z^2}{y^2z}=\frac{1}{x}-\frac{x}{y^2}+\frac{1}{z}-\frac{z}{y^2}$$
$$g"(y)=\frac{3y}{x^3}+\frac{3z}{x^3}>0$$
This shows that $g$ is strictly convex on $[x,z]$, and the statement follows from the fact that $g(x)=g(z)=d^2_s(a+xi,a+yi)$ and that $g(y)=d^2_s(a+xi,a+yi)+d^2_s(a+yi,a+zi)$
\end{proof}


%\textcolor{red}{Menglin, quand il y  des connecteurs autour dúne fraction, c'est mieux si tu écris ``/left/Vert,... /right/Vert'' (avec des backslashs), pareil pour les parenthèse, crochets et accolades:
%$$d_s(\Delta,\Delta')=\left\Vert \frac{(\Delta-\Delta')^2}{\Delta\Delta'}\right\Vert$$}


\begin{lemma}
    Given $\Delta,\Delta'\in \mathcal{D}_n(\mathbb{H}):$ and $\Lambda\in \mathcal{D}^+_n$
    $$d_s(\Lambda\Delta,\Lambda\Delta')=d_s(\Delta,\Delta')$$ 
    $$d_s(-\Delta^{-1},-\Delta'^{-1})=d_s(\Delta,\Delta')$$
\end{lemma}

\begin{lemma}
    Given four diagonal matrices $\Delta,\Delta',D,D'\in \mathcal{D}_n(\mathbb{H}):$
    $$d_s(\Delta+D,\Delta'+D')\leq \max(d_s(\Delta,\Delta'),d_s(D,D'))$$
\end{lemma}
\begin{proof}
    For any $\Delta,\Delta',D,D'\in \mathcal{D}_n(\mathbb{H}):$, there exist $i_0\in [n]$ such that:
    \begin{equation*}
    \begin{split}
    d_s(\Delta+D,\Delta'+D')&=\frac{|\lambda_{i_0}-\Lambda'_{i_0}+D_{i_0}-D'_{i_0}|}{\sqrt{\Im(\Delta_{i_0}+D_{i_0})\Im(\Delta'_{i_0}+D'_{i_0})}}\\
    &\leq \frac{|\lambda_{i_0}-\Lambda'_{i_0}|+|D_{i_0}-D'_{i_0}|^2}{\sqrt{\Im(\Delta_{i_0})\Im(\Delta'_{i_0})}+\sqrt{\Im(D_{i_0})\Im(D'_{i_0})}}\\
    &\leq \max\left( \frac{|\lambda_{i_0}-\Lambda'_{i_0}|}{\sqrt{\Im(\Delta_{i_0})\Im(\Delta'_{i_0})}}, \frac{|D_{i_0}-D'_{i_0}|}{\sqrt{\Im(D_{i_0})\Im(D'_{i_0})}}\right)    
    \end{split}    
    \end{equation*}
\end{proof}

In proving this property we have used the following elementary inequality results.

\begin{lemma}
    Given four positive real numbers $a,b,\alpha,\beta$:
     $$\sqrt{ab}+\sqrt{\alpha \beta}\leq \sqrt{(a+\alpha)(b+\beta)}$$
     $$\frac{a+\alpha}{b+\beta}\leq max(\frac{a}{b},\frac{\alpha}{\beta})$$
\end{lemma}
\begin{proof}
    For the first result, we deduce from the inequality $2\sqrt{ab\alpha\beta}\leq a\beta+b\alpha$:
$$(\sqrt{ab}+\sqrt{\alpha\beta})^2=ab+\alpha\beta+2\sqrt{ab\alpha\beta}\leq ab+\alpha\beta+a\beta+b\alpha$$
For the second result, we simply bound:
$$\frac{a+\alpha}{b+\beta}= \frac{a}{b}\frac{b}{b+\beta}+\frac{\alpha}{\beta}\frac{\beta}{b+\beta}\leq max\left(\frac{a}{b},\frac{\alpha}{\beta}\right).$$
\end{proof}

\begin{definition}
    Given $\lambda>0$, we denote $\mathcal{C}^{\lambda}_s(\mathcal{D}_n(\mathbb{H}))$, the class of functions $f:\mathcal{D}_n(\mathbb{H})\to \mathcal{D}_n(\mathbb{H})$, $\lambda$-Lipschitz for the semi-metric $d_s$; i.e. satisfying for all $D,D'\in \mathcal{D}_n(\mathbb{H}):$
    $$d_s(f(D),f(D'))\leq \lambda d_s(D,D').$$
    When $\lambda<1,$ we say that $f$ is contracting for the semi-metric $d_s$.
\end{definition}


\begin{proposition}
    Given three parameters $\alpha,\lambda,\theta>0$ and two mappings $f\in \mathcal{C}^{\lambda}_s$ and $g\in \mathcal{C}^{\theta}_s,$
\begin{equation*}
    \frac{-1}{f}\in \mathcal{C}^{\lambda}_s,\ \ \ \alpha f\in \mathcal{C}^{\lambda}_s,\ \ \ f\circ g\in \mathcal{C}^{\lambda\theta}_s, \ \ \ f+g \in \mathcal{C}^{\max(\lambda,\theta)}_s
\end{equation*}
    
\end{proposition}


\section{Fixed point theorem for contracting mapping}

The Banach fixed point theorem states that a contracting function on a complete space admits a unique fixed point. The extension of this result to contracting mappings on $\mathcal{D}_n(\mathbb{H})$, for the semi-metric $d_s$, is not obvious: first, because $d_s$ does not verify the triangular inequality and second because the completeness needs to be proven. The completeness is guaranteed by a boundedness condition that we impose on the matrices.

%\begin{theorem}
%    If $f:\mathcal{D}^+_n\to \mathcal{D}^+_n$ is contracting for the semi-distance $d_s$, then it's also contracting for the hyperbolic metric $d$.(\textcolor{red}{wrong without the compactness})
%\end{theorem}
%\begin{proof}
%    Let $c(r)$ be the function $c(r):=cosh^{-1}(\lambda r+1)/cosh^{-1}(r+1)$, then $c(r)<1 \ \ \forall r$ as $cosh^{-1}$ is increasing. In particular, for $r=\frac{1}{2}d_s(\Delta,\Delta')$, we obtain
%    $$cosh^{-1}(\frac{\lambda}{2}d_s(\Delta,\Delta')+1)\leq c_1 cosh^{-1}(\frac{\lambda}{2}d_s(\Delta,\Delta')+1)$$
%    Still by the increasing property we have
%    $$d_H(f(\Delta),f(\Delta'))=cosh^{-1}(\frac{1}{2}d_s(f(\Delta),f(\Delta'))+1)\leq cosh^{-1}(\frac{\lambda}{2}d_s(\Delta,\Delta')+1)\leq c_1 cosh^{-1}(\frac{\lambda}{2}d_s(\Delta,\Delta')+1)$$
%\end{proof}

\begin{theorem}
Given a subset $\mathcal{D}_b$ of $\mathcal{D}_n(\mathbb{H})$ where each diagonal entry has an imaginary part bounded from above and below and a mapping $f:\mathcal{D}_b \to \mathcal{D}_b$ , if it is furthermore contracting for the stable semi-metric $d_s$ on $\mathcal{D}_b$, then there exists a unique fixed point $\Delta^*\in \mathcal{D}_b$ satisfying $\Delta^*=f(\Delta^*)$. 
\end{theorem}
\begin{proof}
Noting $\lambda \in(0,1)$ the Lipschitz constant such that $\forall \Delta,\Delta'\in \mathcal{D}_n(\mathbb{H}), d_s(f(\Delta),f(\Delta'))\leq \lambda d_s(\Delta,\Delta'),$ we show that the sequence $(\Delta^{(k)})_{k\geq 0}$ satisfying:
$$\Delta^{(0)}=I_n, \ \ \ \forall k\geq 1, \Delta^{(k)}=f(\Delta^{(k-1)})$$
is a Cauchy sequence in $\Bar{\mathcal{D}}_n(\mathbb{H})$, where $\Bar{\mathcal{D}}_n(\mathbb{H})\equiv \mathcal{D}_n(\mathbb{H}\bigcup \mathbb{R}).$
$\forall p\in\mathbb{N}$, $\Delta^{(p)}\in \mathcal{D}_b$, i.e. there exists $\delta>0$, such that $|\Im \Delta^{(p)}|\leq \delta.$
We can then bound for any $p\in\mathbb{N}:$
$$\Vert \Delta^{(p+1)}-\Delta^{(p)}\Vert \leq \delta d_s(\Delta^{(p+1)},\Delta^{(p)})\leq \lambda^p \delta d_s(\Delta^{(1)},\Delta^{(0)}).$$
Therefore, thanks to the triangular inequality in $(\mathcal{D}_n(\mathbb{H}),\Vert\cdot\Vert),$ for any $n\in\mathbb{N}:$
\begin{equation*}
\begin{split}
\Vert \Delta^{(p+n)}-\Delta^{(p)}\Vert &\leq \Vert \Delta^{(p+n)}-\Delta^{(p+n-1)}\Vert+\cdots +\Vert\Delta^{(p+1)}-\Delta^{(p)}\Vert \\
&\leq \frac{\delta d_s(\Delta^{(1)},\Delta^{(0))}}{1-\lambda}\lambda^p\to 0.    
\end{split}
\end{equation*}
This allows us to conclude that $(\Delta^{(p)})_{p\in \mathbb{N}}$ is a Cauchy sequence, and therefore it converges to a diagonal matrix $\Delta^{*}\equiv \lim_{p\to\infty} \Delta^{(p)}\in \Bar{\mathcal{D}}_n(\mathbb{H})$ which is a closed thus complete set. But since $\Delta^{(p)}$ has diagonal entries which are bounded from below, we know that $\Delta^*\in \mathcal{D}_b$.
By contractivity of $f$, it is clearly unique.
\end{proof}

\section{Stability of the stable semi-metric towards perturbations}
We have first of all the following elementary inequality result.
\begin{lemma}
    Given theree diagonal matrices $\Gamma^1,\Gamma^2,\Gamma^3\in\mathcal{D}_n(\mathbb{H})$:
    $$\left\Vert \frac{\Gamma^3}{\sqrt{\Im(\Gamma^1)}} \right\Vert\leq \left\Vert \frac{\Gamma^3}{\sqrt{\Im(\Gamma^2)}}(1+d_s(\Im(\Gamma^1),\Im(\Gamma^2))) \right\Vert$$
\end{lemma}
\begin{proof}
    We simply bound:
     \begin{equation*}
    \begin{split}
    \left\Vert \frac{\Gamma^3}{\sqrt{\Im(\Gamma^1)}} \right\Vert 
    &\leq  \left\Vert \frac{\Gamma^3}{\sqrt{\Im(\Gamma^2)}} \right\Vert 
+\left\Vert \frac{\Gamma^3 \left(\sqrt{\Im(\Gamma^2)}-\sqrt{\Im(\Gamma^1)}\right)}{\sqrt{\Im(\Gamma^2)\Im(\Gamma^1)}} \right\Vert \\
    &\leq \left\Vert \frac{\Gamma^3}{\sqrt{\Im(\Gamma^2)}} \right\Vert 
+\left\Vert \frac{\Gamma^3}{\sqrt{\Im(\Gamma^2)}} \right\Vert \left\Vert \frac{\Im(\Gamma^2)-\Im(\Gamma^1)}{\sqrt{\Im(\Gamma^1)}\left(\sqrt{\Im(\Gamma^2)}+\sqrt{\Im(\Gamma^1)}\right)} \right\Vert
    \end{split}
    \end{equation*}
\end{proof}

Next we give the result to bound the distance between a diagonal matrix and the other one which is obtained as a fixed point.
\begin{proposition}
    Given a diagonal matrix $\Gamma\in \mathcal{D}_n(\mathbb{H})$, a mapping $f:\mathcal{D}_n(\mathbb{H})\to \mathcal{D}_n(\mathbb{H})$ $\lambda$ contractive for the semi-metric $d_s$ with the Lipschitz coefficient $\lambda<1$ and admitting the fixed point $\Tilde{\Gamma}=f(\Tilde{\Gamma})$, we have the bound:
    \begin{equation*}
        d_s(\Gamma,\Tilde{\Gamma})\leq \left\Vert \frac{f(\Gamma)-\Gamma}{\sqrt{\Im(\Gamma)\Im(\Tilde{\Gamma})}}\right\Vert/(1-\lambda-\lambda d_s(\Im(\Gamma)),\Im(f(\Gamma))) 
    \end{equation*}
\end{proposition}
\begin{proof}
    Thanks to the above lemma, we can bound:
    \begin{equation*}
    \begin{split}
        d_s(\Gamma,\Tilde{\Gamma})&\leq \left\Vert \frac{\tilde{\Gamma}-f(\Gamma)}{\sqrt{\Im(\Gamma)\Im(\Tilde{\Gamma})}}\right\Vert+\left\Vert \frac{f(\Gamma)-\Gamma}{\sqrt{\Im(\tilde{\Gamma})\Im(\Gamma)}} \right\Vert\\
        &\leq d_s(\tilde{\Gamma},\Gamma)(1+d_s(\Im(\Gamma)),\Im(f(\Gamma))+\left\Vert \frac{f(\Gamma)-\Gamma}{\sqrt{\Im(\tilde{\Gamma})\Im(\Gamma)}} \right\Vert\\
        &\leq \left\Vert \frac{f(\Gamma)-\Gamma}{\sqrt{\Im(\Gamma)\Im(\Tilde{\Gamma})}}\right\Vert/(1-\lambda-\lambda d_s(\Im(\Gamma)),\Im(f(\Gamma))) 
    \end{split}
    \end{equation*}
\end{proof}
\begin{proposition}
    Let us consider a family of mappings$(f^m)_{m\in\mathbb{N}}$ of $\mathcal{D}_{n_m}(\mathbb{H})$, each $f^m$ being $\lambda$-Lipschitz for the semi-metric $d_s$ with $\lambda<1$ and admitting the fixed point $\Tilde{\Gamma}^m=f^m(\Tilde{\Gamma}^m)$ and a family of diagonal matrices $\Gamma^m$. If one assume that $d_s(\Im(\Gamma^m),\Im(f^m(\Gamma^m)))\leq o_{m\to \infty}(1)$, then 
    $$d_s(\Gamma^m,\Tilde{\Gamma}^m)\leq O_{m\to\infty}\left(\left\Vert\frac{f^m(\Gamma^m))-\Gamma^m}{\sqrt{\Im(\Tilde{\Gamma^m})\Im(\Gamma^m)}}\right\Vert\right)$$
\end{proposition}
\begin{proof}
    For $m$ sufficiently big, we have $d_s(\Im(\Gamma^m),\Im(f^m(\Gamma^m)))\leq o(1)\leq \frac{1-\lambda}{2\lambda}$, so we have:
    \begin{equation*}
    \begin{split}
d_s(\Gamma^m,\Tilde{\Gamma}^m)& \leq \left\Vert \frac{f(\Gamma^m)-\Gamma^m}{\sqrt{\Im(\Gamma^m)\Im(\Tilde{\Gamma}^m)}}\right\Vert/(1-\lambda-\lambda d_s(\Im(\Gamma^m)),\Im(f(\Gamma^m))) \\
&\leq \left(\left\Vert\frac{f^m(\Gamma^m))-\Gamma^m}{\sqrt{\Im(\Tilde{\Gamma^m})\Im(\Gamma^m)}}\right\Vert\right)
    \end{split}
    \end{equation*}
\end{proof}

\end{document}
\end{document}
