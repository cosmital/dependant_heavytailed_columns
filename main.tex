\documentclass[ECP, preprint]{ejpecp} % replace ECP by EJP if needed.

\usepackage{amsmath}
\usepackage{amstext}
\usepackage{amsfonts}
\usepackage{amssymb}
\usepackage{graphicx}
\usepackage{bbm}
% \usepackage{CJK}
\usepackage{tikz}
\usepackage{pgfplots}
\usepackage[T2A,T1]{fontenc}
\newcommand\restrict[2]{{#1}\raisebox{-.5ex}{$|$}_{#2}}


\usepackage{calc} 
\newcommand{\op}{ \mathop{ \vphantom{\bigoplus} \mathchoice {\vcenter{\hbox{\resizebox{\widthof{$\displaystyle\bigoplus$}}{!}{$\boxplus$}}}} {\vcenter{\hbox{\resizebox{\widthof{$\bigoplus$}}{!}{$\boxplus$}}}} {\vcenter{\hbox{\resizebox{\widthof{$\scriptstyle\oplus$}}{!}{$\boxplus$}}}} {\vcenter{\hbox{\resizebox{\widthof{$\scriptscriptstyle\oplus$}}{!}{$\boxplus$}}}} }\displaylimits } 

\renewcommand{\arraystretch}{1.5}
\newcommand{\lip}{\mathcal L \textit{ip}}

\newcommand{\un}{\mathbbm{1}}
\newcommand{\gP}{>}
\DeclareMathOperator{\tr}{Tr}
\DeclareMathOperator{\diag}{Diag}
\DeclareMathOperator{\gra}{Gra}
\DeclareMathOperator{\ran}{Ran}
\DeclareMathOperator{\spctr}{Sp}
\DeclareMathOperator{\incr}{inc}
\DeclareMathOperator{\trd}{TrDg}
\DeclareMathOperator{\res}{R}
\DeclareMathOperator{\dimm}{dim}
\DeclareMathOperator{\vect}{Vect}
\DeclareMathOperator{\conv}{Conv}
\DeclareMathOperator{\dom}{Dom}
\DeclareMathOperator{\epi}{Epi}
\DeclareMathOperator{\id}{Id}
\DeclareMathOperator{\inv}{Iv}
\DeclareMathOperator{\cl}{Cl}
\DeclareMathOperator{\intr}{Int}
\DeclareMathOperator{\et}{ and }
\DeclareMathOperator{\ou}{ or }
\DeclareMathOperator{\si}{ if }
\DeclareMathOperator{\into}{ It }
\DeclareMathOperator{\hs}{HS}

% \usepackage[
% backend=biber,
% style=alphabetic,
% sorting=ynt
% ]{biblatex}
% \addbibresource{D:/OneDrive - CUHK-Shenzhen/Documents/Research/biblio.bib}

% \bibliography{D:/OneDrive - CUHK-Shenzhen/Documents/Research/biblio}

\newcommand\script[1]{{\fontfamily{pzc}\fontshape{it}\selectfont#1}}
\SHORTTITLE{Operations \& concentration}

\TITLE{Eigen behaviour of Random matrices with \\ Heavy tailed independent columns}
    %\thanks{Current maintainer of class file is
     % \href{https://vtex.lt}{VTeX, Lithuania}. Please send all queries to
     % \href{mailto:latex-support@vtex.lt}{\texttt{latex-support@vtex.lt}}.}} % \thanks is optional. Insert line breaks with \\

%\DEDICATORY{Dedicated to the memory of ...} % Optional

%%%%%%%%%%%%%%%%%%%%%%%%%%%%%%%%%%%%%%%%%%%%%%%%%%%%%%%%%%%%%%%%%%%
%%                                                               %%
%% Authors (please edit and customize):                          %%
%%                                                               %%
%%%%%%%%%%%%%%%%%%%%%%%%%%%%%%%%%%%%%%%%%%%%%%%%%%%%%%%%%%%%%%%%%%%
% \AUTHORS{%
%   Cosme Louart\footnote{ CUHK (Shenzhen), School of data science. \EMAIL{cosmelouart@cuhk.edu.cn}}
%   \and %% remove this line and below if single author
%   }
\AUTHORS{Authors\footnote{School of Data Science, The Chinese University of Hong Kong (Shenzhen), Shenzhen, China}}
%\and%%
%Romain Couillet\footnote{LIG-lab,
%GIPSA-lab. \EMAIL{romain.couillet@gipsa-lab.grenoble-inp.fr}}}
%% Type \and between all consecutive authors (not only before the last author).
%% Note: you may use \BEMAIL to force a line break before e-mail display.

%% Here is a compact example with two authors with same affiliation
%% \AUTHORS{%
%%  Michael~First\footnote{Some University. \EMAIL{mf,js@uni.edu}
%%  \and
%%  John~Second\footnotemark[2]}%AUTHORS
%% Note: The \footnotemark is the footnote number that you wish to reuse. Here
%% it is [2] (we took into account the footnote generated by \thanks in title).

%%%%%%%%%%%%%%%%%%%%%%%%%%%%%%%%%%%%%%%%%%%%%%%%%%%%%%%%%%%%%%%%%%%
%%                                                               %%
%% Please edit and customize the following items:                %%
%%                                                               %%
%%%%%%%%%%%%%%%%%%%%%%%%%%%%%%%%%%%%%%%%%%%%%%%%%%%%%%%%%%%%%%%%%%%

\KEYWORDS{Random matrix; Heavy tailed concentration; Hanson-Wright inequality} % Separate items with ;

\AMSSUBJ{60-08, 60B20, 62J07} % Edit. Separate items with ;
%\AMSSUBJSECONDARY{FIXME:} % Optional, separate items with ;

\SUBMITTED{} % Edit.
%\ACCEPTED{} % Edit.

%%%%%%%%%%%%%%%%%%%%%%%%%%%%%%%%%%%%%%%%%%%%%%%%%%%%%%%%%%%%%%%%%%%
%%                                                               %%
%% Please uncomment and edit if you have an arXiv ID:            %%
%%                                                               %%
%%%%%%%%%%%%%%%%%%%%%%%%%%%%%%%%%%%%%%%%%%%%%%%%%%%%%%%%%%%%%%%%%%%

%\ARXIVID{2102.08020} % Edit.

\ABSTRACT{Abstract}
\begin{document}
\section{Notations}
Let us introduce the notations $\mathbb R_+\equiv [0,\infty)$, $\mathbb R_+^*\equiv (0,+\infty)$ and $\mathbb H \equiv \{z\in \mathbb C, \Im(z)>0\}$ (the complex half plane).
Given $n,p\in \mathbb N$, $[n] \equiv \{1, \ldots,  n\}$, the entries of a vector $x\in \mathbb C^p$ are generally denoted $x_1, \ldots, x_p$, the columns of a complex matrix $A\in \mathcal{M}_{p,n}$ are denoted $a_1, \ldots, a_n$. Let us denote $\mathcal{M}_{n}$, the set of square matrices $\mathcal{M}_{n,n}$, $S_n$, the set of (possibly non-real) symmetric matrices, $H_n$, the set of Hermitian matrices, $\mathcal O_n$, the set of orthogonal matrices, $\mathcal U_n$, the set of unitary matrices and $D_n$, the set of diagonal matrices. We introduce the natural order relation on $H_n$, given $A,B\in H_n$:
\begin{align*}
     A\leq B
     &&\Longrightarrow&&
     \forall x\in \mathbb C^n:\quad x^*(B-A)x\geq 0.
 \end{align*} 
 % Besides, given a square matrix $A \in \mathcal M_{p}(\mathbb C)$, we denote $|A| = \sqrt{AA^*}\in \mathcal{H}_{p}$.
 Given $x\in \mathbb C^n$, $D = \diag(x) \in D_n$ is the diagonal matrix having the elements $x_1, \ldots,x_n$ on the diagonal then one usually denote $\forall i\in [n]$, $D_i\equiv x_i$. The conjugate transpose of a matrix $M$ is denoted $M^* = \bar{M}^T$. Given a square matrix $A \in \mathcal M_{p}(\mathbb C)$, the spectrum of $A$ is $\spctr(A)$ and we denote $|A| = \sqrt{AA^*}\in \mathcal{H}_{p}$.

 The $\ell_2$ norm on $\mathbb C^p$ is denoted $\|\cdot\|$ ($\|x\| \equiv \sqrt{\sum_{i=1}^p |x_i|^2}$), then the Hilbert-Schmidt norm is denoted $\|\cdot \|_{\hs}$ ($\forall M\in \mathcal{M}_{p,n}$: $\|M\|_{\hs} = \tr(MM^*)$) and the spectral norm is denoted $\|\cdot\|$ ($\|M\| = \sup_{\|x\|=1}\|Mx\|$). Given two normed vector spaces $(E,\|\cdot \|)$ and $(E', \|\cdot \|')$, and a linear mapping $u:E\to F$, the operator norm of $u$ is denoted $\|u\| \equiv \sup_{\|x\|\leq 1}\|u(x)\|'$.



We will set in this paper \textit{quasi-asymptotic} results on random matrices, meaning that we will express convergence results for inequalities or concentration inequalities when important quantities like the number of rows $p$ and the number of columns $n$ converge to $\infty$ or the imaginary part of the complex parameter $z\in \mathbb C$ appearing in the definition of the Stieltjes Transform tends to zero. Just the rate of convergence is relevant, therefore, in order to remove smoothly the constants from the quasi-asymptotic result, we will introduce several notations. Below, the set of indexes $I$ could be thought to be $\mathbb N\times \mathbb N \times \mathbb C$ or even something more elaborate like $\{(p,n,z) \in \mathbb N\times \mathbb N \times \mathbb C, p\leq n, \Im(z)>0\}$ (see Assumption~\ref{ass:n_p_commensurable}).

Given an index set $\Theta$ and two family of parameters $(a_\theta)_{\theta\in \Theta}\in \mathbb R_+Î$ and $(b_\theta)_{\theta\in \Theta}\in \mathbb R_+^\Theta$, we denote: ``$a_\theta\leq O(b_\theta), \  \theta\in \Theta$'' or more simply ``$a\leq O(b)$'' iif there exists a constant $C>0$ such that $\forall \theta\in \Theta$: $a_\theta \leq C b_\theta$ (and we note $a\geq O(b)$ iif $\exists C>0$ such that $\forall \theta\in \Theta$, $a_\theta\geq Cb_\theta$). If $A,B \in \prod_{\theta\in \Theta}H_{n_\theta}$ are two families of Hermitian matrices, $A\leq O(B)$ means that there exists a constant $C>0$ such that: 
\begin{align*}
    \forall \theta \in \Theta:\qquad B_\theta-A_\theta \geq C I_{n_\theta}.
\end{align*}

Following a previous work done in \cite , we will express concentration inequalities with operators which are set valued mappings. An operator $\alpha: \mathbb R \mapsto 2^{\mathbb R}$ is said to be a positive probabilitic operator and we denote $\alpha \in \mathcal M_{\mathbb P_+}$ iif it is maximally decreasing\footnote{Following the monotone operator theory (see for instance~\cite{}), given an operator $\alpha :\mathbb R \mapsto 2^{\mathbb R}$, one denotes $\gra(\alpha) \equiv \{(x,y) \in \mathbb R^2: y\in \alpha(x)\}$, the graph of $\alpha$, $\dom(\alpha) \equiv \{x\in \mathbb R, f(x) \neq \emptyset\}$, the domain of $\alpha$ and $\ran(\alpha) \equiv \{y \in \mathbb R, \exists x\in \dom(\alpha): y \in \alpha(x)\}$ then $\alpha$ is maximally decreasing iif it satisfies the implication $\forall x,y\in \mathbb R^ 2 $: 
\begin{align*}
    \forall (w,z) \in \gra(\alpha): (x-w)(z-y) \geq 0
    &&\Longrightarrow&&
    (x,y) \in \gra(\alpha).
\end{align*}} $\{1\}\subset \ran(\alpha)$ and $\dom(\alpha)\subset \mathbb R_+$. 
Let us consider a family of random variables $(X_\theta)_{\theta\in \Theta}\in \mathbb R^\Theta$ and a family of positive probabilitic operators $(\alpha_\theta)_\theta\in \mathcal M_{\mathbb P_+}^\Theta$.
If there exists some constants $C,c>0$ such that $\forall \theta\in \Theta$:
\begin{align*}
    \forall t\geq 0:\quad \mathbb P(|X_\theta - X_\theta'|\geq t) \leq C\alpha_\theta(ct),
\end{align*}
where $(X_\theta')_{\theta\in \Theta}$ is a family of independent copies of $X_\theta$, $\theta\in \Theta$, then we denote $X \in \alpha$ or if one needs to describe more precisely the dependence on $\Theta$: 
\begin{align*}
    X_\theta \in \alpha_\theta, \theta \in \Theta.
\end{align*}
When there exists a family of deterministic parameters $(\tilde X_\theta)_{\theta\in \Theta}$ such that $\forall \theta\in \Theta$:
\begin{align*}
    \forall t\geq 0:\quad \mathbb P(|X_\theta - \tilde X_\theta|\geq t) \leq C\alpha_\theta(ct),
\end{align*}
for some constants $C,c>0$, one denotes $X\in\tilde X \pm \alpha$ or more simply $X\in O(m) \pm \alpha$, for any $(m_\theta)_{\theta \in \Theta}$ such that $|\tilde X| \leq O(m)$.

We rely on real-valued functional to extend those notations to random vectors. Given a family of normed vector spaces $(E_\theta,\|\cdot\|)_{\theta\in \Theta}$, a family of random vectors $(X_\theta)_\theta\in \prod_{\theta\in \Theta} E_\theta$, the notation ``$X\in \alpha$'' means that there exists some constants $C,c>0$ such that\footnote{With the random variable notations, that means that:
\begin{align*}
    f(X_\theta) \in \alpha_\theta, \qquad \theta\in \Theta, \ f:E_\theta \to \mathbb R, \ 1\text{-Lipschitz}.
\end{align*}} $\forall \theta\in \Theta$ for all $1$-Lipschitz mappings $f: E_\theta \to \mathbb R$:
\begin{align*}
    \forall t\geq 0:\quad \mathbb P(|f(X_\theta) - f(X_\theta')|\geq t) \leq C\alpha_\theta(ct),
\end{align*}
where for all $\theta \in \Theta$, $X_\theta'$ is an independent copy of $X_\theta$. If in addition, we are given a family of deterministic vectors $(\tilde X_\theta)_{\theta \in \Theta} \in \prod_{\theta\in \Theta} E_\theta$ such that exists some constants $C,c>0$ such that\footnote{With previous notations, that means that $X \in \alpha$ and:
\begin{align*}
    u(X_\theta) \in u(\tilde X_\theta) \pm \alpha_\theta, \qquad \theta\in \Theta, \ u:E_\theta \to \mathbb R, \ \text{linear}, \ \|u\| \leq 1.
\end{align*}} $\forall \theta\in \Theta$ for all linear form $u: E_\theta \to \mathbb R$ such that $\|u\|\leq 1$:
\begin{align*}
    \forall t\geq 0:\quad \mathbb P(|u(X_\theta - \tilde X_\theta)|\geq t) \leq C\alpha_\theta(ct).
\end{align*}

We denote $\id$, the identity operator $t\mapsto \{t\}$, then $\sqrt{\id}:t\mapsto \{\sqrt{t}\}$, it satisfies $\dom(\sqrt{\id})\subset \mathbb R_+$.

\section{Setting}
By default the sets of matrices $\mathcal{M}_{p,n}$ (in particular $D_n\subset \mathcal{M}_{n}$), $p,n\in \mathbb N$ are endowed with Hilbert-Schmidt norms $\|\cdot \|_{\hs}$ and the sets of random vectors $\mathbb R^p$, $p\in \mathbb N$ are endowed with the $\ell_2$ norm. 

In what follow, we consider a constant $\gamma>0$ and introduce:
\begin{align*}
    \Theta_{\gamma} \equiv \{(n,p) \in \mathbb N^2, n\geq \gamma p\}.
    % \Theta_{\gamma} \equiv \{(n,p,z) \in \mathbb N^2\times \mathbb H, n\geq \gamma p\}.
\end{align*}
the index set that will direct our quasi-asymptotic results.

Considering a family of random matrices $X = (X_{(n,p)})_{(n,p)\in \Theta_\gamma}$, given $i\in \mathbb N$, let us naturally denote $x_i \equiv (x_i^{(n,p)})_{(n,p)\in \Theta_\gamma, n\geq i}$, the family of the $i^{th}$ column of $X$ and introduce the family of means, of centered and non-centered empirical covariance matrices for all $i\in \mathbb N$:
 % respectively $\mu \equiv (\mu_i^{(n,p)})_{(n,p)\in \Theta_\gamma, i\in [n]}$, $\Sigma \equiv (\Sigma_i^{(n,p)})_{(n,p)\in \Theta_\gamma, i\in [n]}$ and $C \equiv (C_i^{(n,p)})_{(n,p)\in \Theta_\gamma, i\in [n]}$ satisfying $\forall n,p \in \Theta_\gamma$:
\begin{align*}
     \mu_i\equiv \mathbb E[x_i]
     \qquad \Sigma_i \equiv \mathbb E[x_i(x_i)^T].
     \qquad C_i \equiv \Sigma_i - \mu_i(\mu_i)^T.
\end{align*} 
% In what follows, relying on our short notations, we will avoid mentioning the indexes $p,n$ as much as possible (and keeping in mind that $p\leq \gamma n$).

Considering a family of positive probability operators $\alpha \in \mathcal M_{\mathbb P_+}^{\Theta_\gamma}$, we will assume the following properties are satisfied:
\begin{itemize}
    \item for all $(n,p)\in \Theta_\gamma$: $x_i^{(n,p)},\ldots, x_n^{(n,p)}$ are independent,
    \item $X \in \alpha$,
    \item $\sigma_\alpha \equiv \int t\alpha(t) dt \leq \infty$ and $\alpha(\sigma\alpha)\leq O(1)$,
    \item $\|\mu_i\| \leq O(1)$, $i\in [n]$,
    \item $\Sigma_i \geq O(1)$, $i\in [n]$.
\end{itemize}

\begin{remark}\label{rem:comment_on_assumptions}
    \begin{itemize}
        \item Possible alpha...
        \item bound on $\|\mu\|$...
        \item lower bound on $\Sigma$...
    \end{itemize}
\end{remark}



\section{Concentration of the resolvent.}
To study the spectral distribution of $\frac{1}{n}XX^T$:
\begin{align*}
    \nu \equiv \frac{1}{n}\sum_{\lambda \in \spctr(\frac{1}{n}XX^T)} \delta_{\lambda},
 \end{align*} 
 the classical approach is to look at the Stieltjes transform defined for any $z\notin \spctr(\frac{1}{n}XX^T)$ as:
 \begin{align*}
    m(z) \equiv \int \frac{1}{z- \lambda } d\nu(\lambda).
 \end{align*}
 To deduce properties on $\nu$, it is sufficient to study $m(z)$ for $z\in \mathbb C$ such that $\Im(z)\in (0,1]$, we will thus restrict our study to this range to simplify the bounds in the convergence results.
 Introducing the family of resolvents $Q\equiv (Q^z_{n,p})_{(n,p) \in \Theta_\gamma, \Im(z)\in(0,1]}\in \prod_{(p,n)\in \Theta_\gamma, \Im(z)\in(0,1]} \mathcal{M}_{p}$ defined for any $(p,n) \in \Theta_\gamma$, $\forall z\in \mathbb H$ as:
 % \footnote{One classically rather study the matrix $-z Q_z = (\frac{1}{n}XX^ T - I_p)$, but dispite the fact that it is conceptually the same object, $Q^z$ has the advantage to be a $O(\frac{1}{|z|\Im(z)\sqrt{n}}$-Lipschitz transformation of $X$, as it will be seen in bounded from below for all values of $z$ as stated in Proposition~\ref{pro:concentration_resolvente_1}. }:
 \begin{align*}
     Q^z = \left(I_p - \frac{1}{n}XX^T\right)^{-1}
     % Q^z = \left( I_p - \frac{1}{zn}XX^T\right)^{-1}
 \end{align*}
 one will rely on the identity:
 \begin{align*}
    m(z) = \frac{1}{p}\tr(Q^z).
    % m(z) = -\frac{1}{zp}\tr(Q^z).
 \end{align*}

 It is somehow convenient to study simultaneously the so-called ``co-resolvent'' $\check Q$ defined as:
 \begin{align*}
    \check Q = \left( zI_n - \frac{1}{n}X^TX\right)^{-1}\in \prod_{(n,p)\in \Theta_\gamma, \Im(z)\in (0,1]} \mathcal{M}_{p,n}.
    % \check Q^z = \left( I_n - \frac{1}{zn}X^TX\right)^{-1}\in \prod_{(n,p)\in \Theta_\gamma} \mathcal{M}_{p,n}.
 \end{align*}
 To set the concentration of $Q$ and $\check Q$, let us first bound them, it is a trivial and classical result of Random matrix theory that we provide here without proof.

\begin{lemma}\label{lem:Borne_resolvante}
$|Q|, |\check Q| \leq O \left(\frac{1}{\Im z}\right)$.
% \begin{align*}
%   O \left(\kappa_z\right) I_p \leq |Q^z| \leq O \left(\kappa_z\right)I_p&
%   &\text{and}&
%   &O \left(\check \kappa_z\right) I_p \leq |\check Q^z| \leq O \left(\check \kappa_z\right)I_p
% \end{align*}
% (for the classical order relation on hermitian matrices).
\end{lemma}
Let us note that from the identity $Q \frac{1}{n}XX^T = Q - I_p$, one can also bound:
\begin{align}\label{eq:bound_QX}
    \left\Vert\frac{1}{n}Q X\right\Vert \leq \frac{1}{\sqrt n}\sqrt{\left\Vert\frac{1}{n}QXX^TQ\right\Vert} \leq \frac{1}{\sqrt n}\sqrt{\left\Vert Q^ 2 -  Q\right\Vert}\leq O \left( \frac{1}{\sqrt{n}\Im(z)} \right)
\end{align}

\begin{proposition}\label{pro:concentration_resolvente_1}
  $Q^z, \check Q^z \in \alpha(\frac{\cdot}{\Im (z)^2\sqrt n})$.
\end{proposition}
\begin{proof}
  % We could see it as a consequence of Theorem~\ref{the:lipschitz_COncentration_solution_conentrated_equation_phi_concentre_norme_infinie}, applied to the equation
  % \begin{align*}
  %   Q^z = \frac{I_p}{z} + \frac{1}{zn} XX^TQ^z,
  % \end{align*}  
  % but it is probably more simple to see 
  Introducing the mappings $\Phi : \mathcal M_{p,n} \to \mathcal M_{p}$ and $\check\Phi: \mathcal M_{p,n} \to \mathcal M_{n}$ defined as:
  \begin{align*}
     \Phi(M) =  \left(zI_p - \frac{MM^T}{n}\right)^{-1}&
     % \Phi(M) =  \left(I_p - \frac{MM^T}{zn}\right)^{-1}&
     &\text{and}&
     &\check \Phi(M) =  \left(zI_n - \frac{M^TM}{n}\right)^{-1},
     % &\check \Phi(M) =  \left(I_n - \frac{M^TM}{zn}\right)^{-1},
   \end{align*}
   it is sufficient to show that $\Phi$ and $\check \Phi$ are both $O(1/\sqrt n\Im(z)^2)$-Lipschitz (for the Hilbert-Schmidt norm).
    % on $\mathcal M_{n,p} \equiv X(\mathcal A_Q)$\footnote{$\mathcal M_{n,p}^{\mathcal A_Q} \subset \{ M \in \mathcal M_{n,p}, \frac{1}{n}\|MM^T\| \leq \nu+\varepsilon \}$}. 
    For any $M \in \mathcal M_{n,p}$ and any $H \in \mathcal M_{p,n}$, we can bound:
     % $\|M\|\leq (\nu + \varepsilon) \sqrt n \leq O(\sqrt n)$ and:
  \begin{align*}
    \left\Vert \restrict{d\Phi }{M} \cdot H \right\Vert_{\hs} 
    &= \left\Vert \Phi \left(M\right)\frac{1}{n}(MH^T + HM^T) \Phi \left(M\right)\right\Vert_{\hs}
    % &\leq \frac{2}{n}\left\Vert \Phi \left(M\right)MH^T\Phi \left(M\right)\right\Vert_{\hs}
    &\leq O \left( \frac{1}{\Im(z)^2\sqrt n} \right) \|H\|_{\hs},
    % = \frac{1}{\sqrt n} \sqrt{\tr \left( \Phi \left(M\right)MH^T\Phi \left(M\right)\Phi \left(M\right)H^T\Phi \left(M\right) \right)}
    % = \left\Vert \Phi(M)(z\Phi(M) - I_p)\right\Vert 
    % \leq O \left(\frac{\|H\|}{\sqrt n}\right).%
    % &\leq \frac{(\nu + \varepsilon) |z| \|H\|}{d(z,S_\varepsilon)^2 \sqrt n} 
    % \leq O \left(\frac{\|H\|}{\sqrt n}\right).
  \end{align*}
  thanks to lemma~\ref{lem:Borne_resolvante} and \eqref{eq:bound_QX}.
  The same holds for $\check Q^z$.
\end{proof}

We also provide here the expression of the concentration of $QX$ and $X^T\check Q$ that will be useful later.
\begin{lemma}\label{lem:concentration_QX_s_z}
  $Q X =  X^T \check Q  \in \alpha \left( \Im(z)^2 \right)$
  % and $\forall i \in [n]$, $\|\mathbb E_ {\mathcal A_Q}[Qx_i]\| \leq O \left( \frac{|z|}{1+|z|} \right)$. 
\end{lemma}
\begin{proof}
  Let us look at the variations of the mapping $\Psi : \mathcal M_{p,n} \to \mathcal M_{p, n}(\mathbb C)$ defined as:
  \begin{align*}
     \Psi(M) =  \left(zI_p - \frac{MM^T}{n}\right)^{-1} M.
   \end{align*}
   to show the concentration of $Q X  = \Psi(X)$. 
   For all $H, M \in \mathcal M_{n,p}$ (and with the notation $\Phi(M) = \left(zI_p - \frac{MM^T}{n}\right)^{-1} $ given in the proof of Proposition~\ref{pro:concentration_resolvente_1}), let us bound:
  \begin{align*}
    \|\restrict{d\Psi }{M} \cdot H\| 
    & \leq \left\Vert \Psi \left(M\right)\frac{1}{n}(MH^T + HM^T) \Psi \left(M\right) M\right\Vert + \left\Vert \Psi \left(M\right)H\right\Vert
    \leq O \left(\frac{\|H\|_{\hs}}{\Im(z)^2}\right).
  \end{align*}
\end{proof}


\section{A first deterministic equivalent}

% One is often merely working with linear functionals of $Q$, and since an elementary result implies that $Q \in \mathbb E Q \pm \mathcal E_2$, one naturally wants to estimate the expectation $\mathbb E [Q]$.
In this subsection, we provide a first estimator of $\mathbb E [Q]$. 


An efficient approach, developed in particular in \cite{SIL86,SIL95} is to look for a deterministic equivalent of $Q^z$ depending on a deterministic diagonal matrix $\Delta \in \mathbb R^n$ and having the form:
\begin{align}\label{eq:definition_of_tilde_Q}
  \tilde Q^\Delta= \left(  zI_p - \Sigma^{\Delta} \right)^{-1}&
  &\text{where}&
  &\Sigma^{\Delta} \equiv  \frac{1}{n}\sum_{i=1}^n \frac{\Sigma_i}{\Delta_i} = \frac{1}{n} \mathbb E[X \Delta^{-1} X^T].
 \end{align}
 One can then express the difference with the expectation $\mathbb{E} \left[Q^z\right]$ followingly:
 \begin{align*}
  \mathbb{E} \left[Q\right] - \tilde Q^\Delta
  &=\mathbb{E}\left[Q\left(\frac{1}{n}XX^T - \Sigma^{\Delta}\right)\tilde{Q}^\Delta\right]
  =\frac{1}{n}\sum_{i=1}^n  \mathbb{E}\left[Q  \left(x_ix_i^T - \frac{\Sigma_i}{\Delta_i} \right)\tilde{Q}^\Delta\right].
  % &\hspace{0.5cm}+ \frac{1}{n^2}\sum_{i=1}^n \mathbb{E}\left[\left(1+\Delta_i\right)Q_{-i}x_ix_i^TQ \Sigma \tilde Q(\Delta) \right]
\end{align*}
To pursue the estimation of the expectation, one needs to control the dependence between $Q$ and $x_i$. For that purpose, one uses classically the Schur identities:
\begin{align}\label{eq:lien_q_qj_schur}
  % &Q=Q_{-i} +\frac{1}{n}(1+ \frac1nx_i^TQx_i)Q_{-i}x_ix_i^TQ_{-i}&
  % &\text{and}&
  % &Qx_i=(1+ \frac1nx_i^TQx_i)Q_{-i}x_i,
  &Q=Q_{-i} +\frac{1}{n}\frac{Q_{-i}x_ix_i^TQ_{-i}}{1- \frac1{n}x_i^TQ_{-i}x_i}&
  &\text{and}&
  &Qx_i=\frac{Q^z_{-i}x_i}{1- \frac1{n}x_i^TQ^z_{-i}x_i},
\end{align}
for $Q_{-i} = (I_n - \frac{1}{zn} X_{-i}X_{-i}^T)^{-1}$ (recall that $X_{-i} = (x_1,\ldots, x_{i-1}, 0, x_{i+1}, \ldots, x_n) \in \mathcal M_{p,n}$). The Schur identities can be seen as simple consequences to the so called ``resolvent identity'' that can be generalized to any, possibly non commuting, square matrices $A,B \in \mathcal M_p$ with the identity:
\begin{align}\label{eq:resolvent_identity}
   A^{-1} - B^{-1} = A^{-1} (B-A) B^{-1}&
   &\text{or}&
   &A^{-1} + B^{-1} = A^{-1} (A+B) B^{-1}
 \end{align}
 (it suffices to note that $A(A^{-1} + A^{-1} (B-A) B^{-1})B = I_p$).

% The central quantity of our estimation is
Introducing the notation:
\begin{align*}
   \Lambda \equiv \diag_{1\leq i \leq n} \left( 1 - \frac1nx_i^TQ_{-i}x_i \right) \in \prod_{(n,p)\in \Theta_\gamma, \Im(z)\in (0,1]} \mathcal D_n,
   % &\text{satisfying: } \ \ \forall i \in [n]: \ Q x_i = \frac{1}{\Lambda_i}Q_{-i} x_i,
 \end{align*} 
 one has the identity $Q x_i = \frac{1}{\Lambda_i}Q_{-i} x_i$.
It is then possible to express:
 % thanks to the independence between $Q_{-i}$ and $x_i$:
\begin{align}\label{eq:definition_epsilon_1_2}
\mathbb{E} \left[Q\right] - \tilde Q^\Delta 
    &= \frac{1}{n}\sum_{i=1}^n  \mathbb{E}\left[Q_{-i} \left(\frac{x_ix_i^T}{\Lambda_i} - \frac{\Sigma_i}{\Delta_i}\right)\tilde{Q}^\Delta\right] 
    + \frac{1}{n}\sum_{i=1}^n \frac{1}{\Delta_i}\mathbb{E}\left[(Q_{-i} - Q) \Sigma_i \tilde Q^\Delta\right]
  % &= \varepsilon_1 + \varepsilon_2 \\
  % &= \varepsilon_1 + \delta_1 + \delta_2 +\varepsilon_2 \\
  % \text{with :} \ \ \ &\left\{
  % \begin{aligned}
  %   &\varepsilon_1 = \frac{1}{n} \mathbb{E}\left[\sum_{i = 1}^nQ_{-i}x_i \left(\frac{\Delta_i - \Lambda_i}{\Lambda_i\Delta_i}\right)x_i^T \tilde{Q}^\Delta \right]
  %    % = \frac{1}{n} \mathbb{E}\left[QX \left(\frac{\Delta - \Lambda}{z\Delta}\right)X^T \tilde{Q}^\Delta \right] \\
  %   % &\delta_1 = \frac{1}{n}\sum_{i=1}^n \mathbb{E}\left[Q_{-i} \left(\frac{x_ix_i^T - \mathbb{E}[x_ix_i^T] }{\Delta_i}\right)\tilde{Q}^\Delta \right] \\
  %   % &\delta_2 = \frac{1}{n}\sum_{i=1}^n \mathbb{E}\left[Q_{-i} \left(\frac{\mathbb{E}[x_ix_i^T] - \Sigma_i}{\Delta_i}\right)\tilde{Q}^\Delta \right] \\
  %   & \varepsilon_2 =  -\frac{1}{zn^2} \sum_{i=1}^n \frac{1}{\Delta_i}\mathbb{E}\left[Qx_i x_i^TQ_{-i}\Sigma_i\tilde{Q}^\Delta\right],
  % \end{aligned}
  % \right. \nonumber
\end{align}
where we recall that $Q - Q_{-i} = \frac{1}{n}Qx_i x_i^TQ_{-i}$.
% \begin{remark}\label{rem:lambda_pas_Q}
% Since $\Lambda$ is appearing on the denominator, one might have been tempted to rather study $\frac{1}{\Lambda}$ and rely on the identity:
% \begin{align}\label{eq:link_lambda_cQ}
%    \frac{1}{\Lambda^z}
%    =\diag_{i\in [n]}\left(\frac{1}{1- \frac1{n}x_i^TQ_{-i}x_i}\right)
%   = I_n + \frac{1}{n}\diag(X^TQX) = \diag(\check Q).
% \end{align}
% Taking $\Delta = \frac{1}{\diag(\check Q)}$ indeed provides a first deterministic equivalent, however, it does not combines well with the fixed point equation on which will rely our second deterministic equivalent.
% % to introduce in next section the fixed point equation 
% % to devise the second deterministic equivalent, one relies on the approximation (that will rigorously set in next section):
% % \begin{align*}
% %     \mathbb E
% % \end{align*}
% \end{remark}

From this decomposition, one is enticed into choosing, in a first step $\Delta = \mathbb E[\Lambda] \in \mathcal D_n(\mathbb C)$ so that $\varepsilon_1$ would be small. 
% Introducing:
% \begin{align*}
%     \hat \Lambda^- \equiv \mathbb E[\Lambda],
% \end{align*}
% one can devise our first determinitic equivalent.

\begin{proposition}\label{pro:first_first_det_eq}
    $\|Q - \tilde Q^{\mathbb E[\Lambda]}\|_{\hs}\leq O \left( \frac{1}{\Im(z)^4\sqrt{n}} \right)$
\end{proposition}
The proof of this proposition relies on the next four preliminary lemmas.
\begin{lemma}\label{lem:lower_bound_lambda}
     $\Lambda \geq O \left( \frac{1}{\Im(z)} \right)$.
 \end{lemma} 
 \begin{proof}
    It is a simple consequence of Lemma~\ref{lem:Borne_resolvante} and the identity:
     \begin{align}\label{eq:link_lambda_cQ}
   \frac{1}{\Lambda^z}
   =\diag_{i\in [n]}\left(\frac{1}{1- \frac1{n}x_i^TQ_{-i}x_i}\right)
  = I_n + \frac{1}{n}\diag(X^TQX) = \diag(\check Q).
\end{align}
 \end{proof}
\begin{lemma}\label{lem:concentration_Q_m_i_autour_Q}
    $\|\mathbb E[Q_{-i}]- \mathbb E[Q] \| \leq O\left( \frac{1}{n\Im(z)^2} \right)$.
\end{lemma}
Note from \cite{} that this lemma implies in particular from Proposition~\ref{pro:concentration_resolvente_1}  that $Q_{-i}\in \mathbb E[Q] \pm \alpha \circ \left( \Im(z)^2\sqrt n \ \id \right)$.
\begin{proof}
    % One already knows from Proposition~\ref{pro:concentration_resolvente_1} that $Q_{-i}\in \alpha \left( \Im(z)^2\sqrt n \ \cdot \right)$ ($Q_{-i}$ satisfies the same hypotheses as $Q$). 
    Let us bound for any deterministic vector $u\in \mathbb C$:
    \begin{align*}
        |u^*(\mathbb E[Q_{-i}] - \mathbb E[Q])u| 
        &= \frac{1}{n} \left\vert \mathbb E \left[ \frac{u^*Q_{-i}x_ix_i^* Q_{-i}u}{\Lambda_i} \right] \right\vert\\
        &\leq  \mathbb E[u^*Q_{-i}\Sigma_iQ_{-i}u] O \left( \frac{1}{n\Im(z)} \right)
        \leq O\left( \frac{1}{n\Im(z)^2} \right)
        % \leq \frac{\sqrt p}{n} \frac{1}{\Im(z)^2}\|A\|_{\hs}.
    \end{align*}
    thanks to Lemmas~\ref{lem:Borne_resolvante} and~\ref{lem:lower_bound_lambda}
    % That implies that $\|\mathbb E[Q_{-i}] - \mathbb E[Q]\|\leq O(\frac{1}{ n \Im(z)^2})$ and therefore $\|\mathbb E[Q_{-i}] - \mathbb E[Q]\|_{\hs}\leq O(\frac{1}{\sqrt n \Im(z)^2})$, one can then conclude with \cite{}. 
\end{proof}
\begin{lemma}\label{lem:bound_Q_tilde}
    Given $\Delta \in \mathcal D_n$, $\|\tilde Q^{\Delta}\|\leq O \left( \frac{1}{\min_{i\in [n]}(\Im(z\Delta_i))} \right)$.
\end{lemma}
\begin{proof}
    \textcolor{red}{TODO}
\end{proof}
\begin{lemma}\label{lem:zlambda_bound}
    $\Im(z\Lambda) \leq \Im(z)$.
\end{lemma}
One can deduce directly from Lemmas~\ref{lem:bound_Q_tilde} and~\ref{lem:zlambda_bound} that $\|\tilde Q^{\mathbb E[\Lambda]}\|\leq O \left( \frac{1}{\Im(z)} \right)$.
\begin{proof}
    Let us compute:
    \begin{align*}
        \Im(z\Lambda)
        = \Im(z) - \frac{1}{n}\tr \left( \Sigma_i (Q-\bar Q) \right)
        = \Im(z) - \frac{1}{n}\tr \left( \Sigma_i Q (\bar zI_p- zI_p)\bar Q \right)
        \geq \Im(z),
    \end{align*}
    since $\tr \left( \Sigma_i Q\bar Q \right)\geq 0$.
\end{proof}
\begin{proof}[proof of Proposition~\ref{pro:first_first_det_eq}]
    To prove our bound let us consider $A\in \mathcal M_p$, and start with the first component of \eqref{eq:definition_epsilon_1_2}:
    \begin{align*}
        &\frac{1}{n}\sum_{i=1}^n  \mathbb{E}\left[\tr  \left( AQ_{-i} \left(\frac{x_ix_i^T}{\Lambda_i} - \frac{\Sigma_i}{\mathbb E[\Lambda_i]}\right)\tilde{Q}^{\mathbb E[\Lambda]} \right)\right]\\
        &\hspace{1cm}= \frac{1}{n}\sum_{i=1}^n  \mathbb{E}\left[x_i^T\tilde{Q}^{\mathbb E[\Lambda]}AQ_{-i} x_i \left(\frac{1}{\Lambda_i} -  \frac{1}{\mathbb E \left[\Lambda_i \right]}\right)\right]\\
        &\hspace{1cm}= \frac{1}{n}\sum_{i=1}^n  \mathbb{E}\left[x_i^T\tilde{Q}^{\mathbb E[\Lambda]}AQ x_i \left(\frac{\mathbb E \left[\Lambda_i \right] - \Lambda_i}{\mathbb E \left[\Lambda_i \right]}\right)\right]\\
        &\hspace{1cm}= \frac{1}{n}\sum_{i=1}^n \frac{1}{\mathbb E[\Lambda_i]} \mathbb{E}\left[\left( x_i^T\tilde{Q}^{\mathbb E[\Lambda]}AQ x_i - \mathbb E \left[ x_i^T\tilde{Q}^{\mathbb E[\Lambda]}AQ x_i \right] \right) \right]\\
        &\hspace{1cm}= \frac{1}{n} \mathbb{E}\left[\tr\left( \frac{1}{\mathbb E[\Lambda]}X^T\tilde{Q}^{\mathbb E[\Lambda]}AQ X\right) - \mathbb E \left[ \tr\left( \frac{1}{\mathbb E[\Lambda]}X^T\tilde{Q}^{\mathbb E[\Lambda]}AQ X\right) \right] \right]
        \leq O \left( \frac{1}{\sqrt{n}\Im(z)^4} \right),
        % \leq O \left( \frac{\|\tilde Q^{\mathbb E[\Lambda]}\|}{\sqrt{n}\Im(z)^3} \right),
    \end{align*}
    thanks to the matricial, heavy-tailed form of Hanson-Wright result ($\sigma_\alpha \int t\alpha \leq \infty$ and $\alpha(\sigma_\alpha)\geq O(1)$) applied to:
    \begin{itemize}
         \item  the concentration $\tilde Q^{\mathbb E[\Lambda]}X\in \alpha $ (by hypothesis on $X$ and thanks to the bound provided by Lemmas~\ref{lem:bound_Q_tilde} and~\ref{lem:zlambda_bound}),
         \item $QX \in \alpha (\frac{1}{\Im(z)^2})$ (see Lemma~\ref{lem:concentration_QX_s_z}),
         \item $\|\frac{1}{\mathbb E[\Lambda]}_{\hs}\leq O \left( \frac{\sqrt{n}}{\Im(z)} \right)$ (see Lemma~\ref{lem:lower_bound_lambda}).
     \end{itemize}

    The second component of~\eqref{eq:definition_epsilon_1_2} is simply bounded thanks to Lemma~\ref{lem:concentration_Q_m_i_autour_Q} that implies:
     % with an application of Cauchy-Schwarz inequality:
    \begin{align*}
        \mathbb{E}\left[\tr \left( A(Q_{-i} - Q) \Sigma_i \tilde Q^{\mathbb E[\Lambda]} \right)\right]
        \leq O \left(  \|\mathbb E[Q_{-i}]- \mathbb E[Q] \| \frac{\sqrt{p}}{\Im(z)} \right)
        % &\hspace{1cm} \leq \frac{1}{n^2}\sqrt{\mathbb{E}\left[\sum_{i=1}^n  \frac{1}{{\mathbb E[\Lambda_i]^2}}\tr \left(AQ_{-i}x_ix_i^T Q_{-i}\right)\right]} \sqrt{\mathbb{E}\left[\sum_{i=1}^n  \tr \left(A\tilde Q^{\mathbb E[\Lambda]}\Sigma_iQx_ix_i^T Q \Sigma_i \tilde Q^{\mathbb E[\Lambda]^2}\right)\right]}\\
        % &\hspace{1cm} = \frac{1}{n^{\frac{3}{2}}}\sqrt{\mathbb{E}\left[\sum_{i=1}^n  \frac{1}{{\mathbb E[\Lambda_i]^2}}\tr \left(AQ_{-i}\Sigma_i Q_{-i}\right)\right]} \sqrt{\mathbb{E}\left[\frac{1}{n} \tr \left(A\tilde Q^{\mathbb E[\Lambda]}\Sigma_iQXX Q \Sigma_i \tilde Q^{\mathbb E[\Lambda]}\right)\right]}
        \leq O \left( \frac{1}{\sqrt n\Im(z)^ 2} \right) 
    \end{align*}
    Combining those two bounds with \eqref{eq:definition_epsilon_1_2}, one obtains the result of the proposition.
    % \begin{align*}
    %     \|Q - \tilde Q^{\mathbb E[\Lambda]}\|_{\hs}\leq O \left( \frac{\|\tilde Q^{\mathbb E[\Lambda]}\|}{\Im(z)^3\sqrt{n}} \right),
    % \end{align*}
    % and in particular:
    % \begin{align*}
    %     \|\tilde Q^{\mathbb E[\Lambda]}\| \leq \|Q - \tilde Q^{\mathbb E[\Lambda]}\|_{\hs} + \|Q\|\leq O \left( \frac{\|\tilde Q^{\mathbb E[\Lambda]}\|}{\Im(z)^3\sqrt{n}} \right) + O \left( \frac{1}{\Im(z)} \right).
    % \end{align*}
    % \textcolor{red}{That does not implies $\|\tilde Q^{\mathbb E[\Lambda]}\|\leq O \left( \frac{\|\tilde Q^{\mathbb E[\Lambda]}\|}{\Im(z)^3} \right)$} from which one can deduce the result of the Proposition
\end{proof}







% To prepare the e of the next deterministic equivallent, we will see that i
In next section, we will see that it is more convenient to work with the deterministic diagonal matrix
\begin{align*}
     \hat \Lambda \equiv
 \diag \left( 1 - \frac{1}{n} \tr(\Sigma_i\mathbb E[Q]) \right)_{1\leq i \leq n} \in \mathcal D_n(\mathbb C),
 \end{align*} 
 which is close to $\mathbb E[\Lambda]$ thanks to Lemma~\ref{lem:concentration_Q_m_i_autour_Q}.

\begin{proposition}\label{pro:Qt1_Qt2}
    $\|\tilde Q^{\mathbb E[\Lambda]} - \tilde Q^{\hat \Lambda}\|\leq O \left( \frac{1}{\Im(z)^6  n} \right)$.
\end{proposition}
\begin{proof}
    Let us bound for any deterministic vector $u\in \mathbb C^p$:
    \begin{align*}
        \left\vert u^* (\tilde Q^{\mathbb E[\Lambda]} - \tilde Q^{\hat \Lambda})u \right\vert
        = \frac{1}{n}\sum_{i=1}^n\left\vert u^* \tilde Q^{\mathbb E[\Lambda]}\Sigma_i  \tilde Q^{\hat \Lambda} u  \right\vert \left\vert \frac{\check \Lambda_i - \mathbb E[\Lambda_i]}{\check \Lambda_i \mathbb E[\Lambda_i]} \right\vert.
    \end{align*}
    One can then directly conclude thanks to the bounds given in Lemmas~\ref{lem:Borne_resolvante} and~\ref{lem:bound_Q_tilde} and the bound:
    \begin{align*}
        |\check \Lambda_i - \mathbb E[\Lambda_i]| = \frac{1}{n}\left\vert \tr(\Sigma_i (Q - Q_{-i})) \right\vert\leq O \left( \frac{1}{\Im(z)^2 n} \right).
    \end{align*}
\end{proof}
% \begin{lemma}\label{lem:lower_bound_lambda}
%      $\Lambda, \check \Lambda \geq O \left( \frac{1}{\Im(z)} \right)$.
%  \end{lemma} 
%  \begin{proof}
%     It is a simple consequence of Lemma~\ref{lem:Borne_resolvante} and the identity:
%      \begin{align}\label{eq:link_lambda_cQ}
%    \frac{1}{\Lambda^z}
%    =\diag_{i\in [n]}\left(\frac{1}{1- \frac1{n}x_i^TQ_{-i}x_i}\right)
%   = I_n + \frac{1}{n}\diag(X^TQX) = \diag(\check Q).
% \end{align}
%  \end{proof}
%  % Since one isbasical
% % One can then set the concentration of $\lambda$ around $\check \Lambda$
% \begin{lemma}\label{lem:Concentration_lambda}
% $\Lambda \in \hat \Lambda \pm \alpha \circ \left( \sqrt{\sqrt n\Im(z)^2 \id} \right)$ (in our regime, $\Im(z)\leq 1$).
%   % \begin{align*}
%   %   (\Lambda \ | \ \mathcal A_Q) \in \mathbb E [\Lambda] \pm \mathcal E_2(|\kappa_z|/\sqrt n)&
%   %   &\text{in} \ \left( \mathcal D_n(\mathbb C), \|\cdot \| \right).
%   % % &\text{in} \ \ (\mathcal D_n(\mathbb C), \| \cdot \|_F).
%   % \end{align*}
%   % and .
% \end{lemma}
% To set this lemma, one first need to justify the fact that one can replace $Q$ with $Q_{-i}$.
% \begin{lemma}\label{lem:conc_uQx}
%     Given a vector $u\in \mathbb C^p$ such that $\|u\|\leq 1$:
%     \begin{align*}
%         u^*Qx_i, u^*Q_{-1}x_i \in O \left( \frac{1}{\Im(z)^2} \right) \pm \alpha \circ \left(\Im(z)^2 \id\right).
%     \end{align*}
% \end{lemma}
% \begin{proof}
%   % Thus, under $\mathcal A_Q$, $\Psi$ is $O(\kappa_z\check\kappa_z)$-Lipschitz (for the Frobenius norm) and therefore $QX \propto \mathcal E_2(\kappa_z\check\kappa_z)$. 
%   The concentration of $u^*Qx_i$ is just a consequence of the concentration of $QX$ given in Lemma~\ref{lem:concentration_QX_s_z}. To show the concentration of $u^*Q_{-i}x_i$ let us start with the decomposition:
%   \begin{align*}
%       |u^*Q_{-i}x_i - u^*\mathbb E[Q_{-i}x_i]|
%       \leq |u^*Q_{-i}(x_i - \mu_i)| + |u^*(Q_{-i} - \mathbb E[Q_{-i}])\mu_i|
%   \end{align*}
%   then one can bound:
%   \begin{align*}
%       \mathbb P \left( |u^*Q_{-i}(x_i - \mu_i)|\geq t \right)\leq C\mathbb E \left[  \alpha \left( \frac{ct}{\|u^*Q_{-i}\|} \right) \right] \leq C\alpha \left( c\Im(z) t \right),
%   \end{align*}

%   It is elementary to bound:
%   \begin{align*}
%       |\mathbb E[u^ *Q_{-i}x_i]| = |u^ * \mathbb E[Q_{-i}]\mu_i| \leq O \left( \frac{1}{\Im(z)} \right).
%   \end{align*}

%   We then rely on this bound to control $ |\mathbb E[u^*Q x_i]|$. Schur identities~\eqref{eq:lien_q_qj_schur} allows to bound:
%   \begin{align*}
%     \left\vert \mathbb E[u^*Q x_i] \right\vert
%     = \left\vert \mathbb E \left[ \frac{ u^*Q_{-i} x_i}{\Lambda_i} \right] \right\vert
%     &\leq \sqrt{\mathbb E \left[ \frac{u^*Q_{-i} x_ix_i^TQ_{-i} u}{|\Lambda_i|^2} \right]}\\
%     &\leq O \left( \frac{1}{\Im(z)} \right) \sqrt{\mathbb E \left[ u^*Q_{-i} \Sigma_i Q_{-i} u \right]} \leq O \left( \frac{1}{\Im(z)^2} \right)
%     % \leq \mathbb E[ |u^TQ_{-i} x_i|]\frac{1}{\Im(z)} \leq O(\kappa_z\check \kappa_z),
%   \end{align*}
%   indeed, $\frac{1}{|\Lambda_i|^2}\leq O (\frac{1}{\Im(z)^2})$ thanks to to Lemma~\ref{lem:Borne_resolvante} combined with \eqref{eq:link_lambda_cQ}.
% \end{proof}

% \begin{proof}[Proof of Lemma~\ref{lem:Concentration_lambda}]
%   Let us bound:
%   \begin{align*}
%     \left\vert \Lambda - \hat \Lambda \right\vert
%     \leq \frac{1}{n}\left\vert x_i^TQ_{-i}x_i - \tr(\Sigma_i Q_{-i}) \right\vert + \frac{1}{n}\left\vert \tr(\Sigma_i (Q_{-i}-\mathbb E[Q])) \right\vert.
%   \end{align*}
%   The Hanson-Wright inequality provided in \cite provides the existence of constants $C,C',c,c'>0$ such that one has for any $(n,p) \in \Theta_\gamma$, any $z\in $ the concentration:
%   \begin{align*}
%       \mathbb P \left( \frac{1}{n}\left\vert x_i^TQ_{-i}x_i - \tr(\Sigma_i Q_{-i}) \right\vert\geq t \right)
%       &= C\mathbb E \left[ \alpha \left( \frac{cnt}{\|Q\|_{\hs}} \right) + \alpha \left( \sqrt{\frac{cnt}{\|Q\|}} \right) \right] \\
%       &\leq C' \alpha \left( \sqrt{c\sqrt n\Im(z)t} \right)
%   \end{align*}
%   since $p\leq O(n)$, $\|Q\|_{\hs}\leq \sqrt p \|Q\|\leq \frac{\sqrt n}{\Im(z)}$ and $\alpha \leq C\alpha(c\sqrt{\id})$ for some constants $C,c>0$.
%   Lemma~\ref{lem:concentration_Q_m_i_autour_Q} and the fact that, in our regime ($p\leq \gamma n$, $\Im(z) \in (0,1]$) $\Im(z)^2 \leq \Im(z)$, then allow us to conclude. 
% \end{proof}


% We can now prove the main result of this subsections that allows us to set that $\tilde Q^{\hat \Lambda}$ is a deterministic equivalent of $Q$.
% \begin{proposition}\label{pro:borne_EQ_m_tQ}
%   % There exist a constant $C>0$ such that:
%   Given $z\in \mathbb C \setminus S_{-0}^\varepsilon$:
%   \begin{align*}
%      \left\Vert \tilde Q^{\hat \Lambda}\right\Vert \leq O \left( \frac{1}{\Im(z)} \right)&
%     % O(1) \leq \tilde Q^{z} \left(\frac{I_n}{I_n - \Delta}\right) \leq O(1)&
%     &\text{and}&
%     &\left\Vert \mathbb{E}[Q] - \tilde Q^{\hat \Lambda} \right\Vert_F \leq  O \left(\frac{1}{\Im(z)^2\sqrt n} \right).
%   \end{align*}
% \end{proposition} 

%  \begin{proof}[Proof of Proposition~\ref{pro:borne_EQ_m_tQ}]
%  Let us note for simplicity $\kappa_{\tilde Q} \equiv \|\tilde Q^{\hat \Lambda}\|$.
%   % With the notation introduced in \eqref{eq:definition_epsilon_1_2}, note that we have to bound $\|\mathbb E[\varepsilon_1] \|_F$ and $\|\mathbb E[\varepsilon_2]\|_F$. 
%   Looking at decomposition \eqref{eq:definition_epsilon_1_2}, considering a  matrix $A\in \mathcal{M}_p(\mathbb C)$ satisfying $\|A\|_F \leq 1$, we can start with the bound:% thanks to Proposition~\ref{pro:estimation_XDY}:
%   \begin{align*}
%      |\tr(A\varepsilon_1)| 
%      &= \left\vert \frac{1}{n} \mathbb{E}\left[\sum_{i = 1}^n\tr \left( AQ_{-i}x_i \left(\frac{\hat \Lambda_i - \Lambda_i}{\Lambda_i\hat \Lambda_i}\right)x_i^T \tilde{Q}^{\hat \Lambda} \right) \right]\right\vert\\
%      &= \left\vert \frac{1}{n} \sum_{i = 1}^n \frac{1}{\hat \Lambda_i} \mathbb{E}\left[ x_i^T \tilde{Q}^{\hat \Lambda}AQx_i \left(\hat \Lambda_i - \mathbb E[\Lambda_i] + \mathbb E[\Lambda_i] -\Lambda_i\right) \right]\right\vert\\
%      &=  \frac{1}{n} \sum_{i = 1}^n \frac{1}{\hat \Lambda_i} \left\vert\mathbb{E}\left[ x_i^T \tilde{Q}^{\hat \Lambda}AQx_i \right]\left(\hat \Lambda_i - \mathbb E[\Lambda_i]\right) \right\vert\\
%      &\hspace{0.5cm} + \left\vert\mathbb{E}\left[ \left( x_i^T \tilde{Q}^{\hat \Lambda}AQx_i - \mathbb E[x_i^T \tilde{Q}^{\hat \Lambda}AQx_i]\right) \Lambda_i]\right) \right]\right\vert\\
%    \end{align*}
%    obtained with the bound $\frac{1}{\hat \Lambda} \leq O(\frac{\check \kappa_z}{|z|})\leq O(1)$ given by Lemma~\ref{lem:Borne_Lambda_hat} and applying Proposition~\ref{pro:estimation_XDY} with the hypotheses:
%    \begin{itemize}
%       \item $X \ | \ \mathcal A_Q \propto \mathcal E_2$ and $\|\mathbb E[x_i]\| \leq O(1)$ Assumption~\ref{ass:concentration_X},
%       \item $QX \ | \ \mathcal A_Q \propto \mathcal E_2(\kappa_z\check\kappa_z) $ and $\|\mathbb E[Qx_i]\|\leq O(\kappa_z\check\kappa_z)$ given by Lemma~\ref{lem:concentration_QX_s_z},
%       \item $ \Lambda \ | \ \mathcal A_Q\in \mathbb E[\Lambda] \pm \mathcal E_2 \left(\frac{\kappa_z}{\sqrt n}\right)$ in $(\mathcal{D}_{n}, \| \cdot \|)$ given by Lemma~\ref{lem:Concentration_lambda},
%       \item $\|  \mathbb E[\Lambda] - \hat \Lambda\|_F \leq O(\kappa_z/\sqrt n)$ thanks to Lemma~\ref{lem:lambda_lambda_hat_proche}
%       % \item $|\hat \Lambda_i| \geq O(|z|/\check \kappa_z)$ given by Lemma~\ref{lem:Borne_Lambda_hat},
%     \end{itemize}
%     % and bounding thanks to .
%   Second, let us bound thanks to Cauchy-Schwarz inequality:
%   \begin{align*}
%     \left\vert \mathbb \tr(A\varepsilon_2) \right\vert
%     &\leq\sqrt{\frac{1}{|z|^2n^2} \mathbb{E}\left[\tr \left(AQ X |\hat \Lambda_n |^{-2} X^T\bar Q \bar A^T\right) \right]}\\
%     &\hspace{1cm}\cdot \sqrt{\frac{1}{n^2}\sum_{i=1}^n \mathbb{E}\left[\tr \left(\tilde{Q}^{\hat \Lambda}\Sigma_ i Q_{-i}x_ix_i^T Q_{-i} \Sigma_i\bar{\tilde{Q}}^{\hat \Lambda}\right)\right]} \\
%     &\leq O \left(\frac{\kappa^4_z \check \kappa_z^4}{|z|^2}\sqrt{\frac{\|A\|_F^2 }{ n}\frac{ \sup_{i\in[n]}\tr \left(\Sigma_i^3\right) \kappa_{\tilde Q}^2 }{ n}}\right) \leq O \left(\frac{\kappa_{\tilde Q}}{\sqrt n}\right)
%     \end{align*}
%     thanks to the bounds provided by our assumptions, and Lemmas~\ref{lem:Borne_resolvante},~\ref{lem:Borne_Lambda_hat} and~\ref{lem:concentration_QX_s_z}. 

%     Taking the supremum on $A \in \mathcal M_{p,n}(\mathbb C)$ and putting the bounds on $\left\Vert \varepsilon_1 \right\Vert_F$, $\left\Vert \varepsilon_2 \right\Vert_F$ together, we obtain:
%     \begin{align*}
%       \left\Vert \mathbb{E} \left[Q\right] - \tilde Q^{\hat \Lambda} \right\Vert_ F \leq O \left(\frac{\kappa_{\tilde Q}}{\sqrt{n}}\right)
%     \end{align*}
%     So, in particular $\kappa_{\tilde Q} \equiv\left\Vert \tilde Q^{\hat \Lambda} \right\Vert \leq  \left\Vert \mathbb{E} \left[Q\right] \right\Vert + O \left(\frac{\kappa_{\tilde Q}}{\sqrt{n}}\right)$,
%     % \begin{align*}
%     %   \kappa_{\tilde Q} \equiv\left\Vert \tilde Q^{\hat \Lambda} \right\Vert \leq  \left\Vert \mathbb{E} \left[Q\right] \right\Vert + O \left(\frac{\kappa_{\tilde Q}}{\sqrt{n}}\right),
%     % \end{align*}
%     which implies that $\kappa_{\tilde Q} \leq O(\kappa_z)$ as $\Vert \mathbb{E} \left[Q\right] \Vert$ since $\frac{1}{\sqrt n} = o(1)$. We obtain then directly the second bound of the proposition.
%     % We obtain then directly $\left\Vert \mathbb{E}Q - \tilde Q^{z} \left(\frac{I_n}{I_n - \Delta}\right) \right\Vert_ F \leq O \left(N_{\tilde Q}\sqrt{\frac{\log n}{n}}\right)$ and $\tilde Q^{z} \left(\frac{I_n}{I_n - \Delta}\right) \geq \mathbb{E}Q  + O \left(N_{\tilde Q}\sqrt{\frac{\log n}{n}}\right)$.
% \end{proof}
\section{A second deterministic equivalent}
\appendix
\section{The semi-metric and Lipschitz mapping}
We introduce the semi-metric $d_s$ on $\mathcal{D}_n(\mathbb{H})=\{D\in \mathcal{D}_n,\forall i \in [n],\Im{D}_i>0\}$:
$$d_s(\Delta,\Delta')=\sup_{1\leq i\leq n}\frac{|\Delta-\Delta'|}{\sqrt{\Im(\Delta)\Im(\Delta')}}$$

The distance $d_s$ is not a metric because it does not satisfy the triangular inequality, see the following counter-example:
$$d_s(4i,i)=\frac{3}{2}>\frac{1}{\sqrt{2}}+\frac{1}{\sqrt{2}}=d_s(4i,2i)+d_s(2i,1i)$$

Indeed, one has the counter-triangular inequality when certain conditions are met:
\begin{lemma}
    Given $x,y,z\in \mathbb{R}$, $x<y<z$ implies that:
    $$d^2_s(a+xi,a+zi)>d^2_s(a+xi,a+yi)+d^2_s(a+yi,a+zi)$$
\end{lemma}
\begin{proof}
    Here we construct the function
    $$g:y\to \frac{(y-x)^2}{xy}+\frac{(z-y)^2}{yz}$$
and we differentiate it twice to get:
$$g'(y)=\frac{y^2-x^2}{xy^2}+\frac{y^2-z^2}{y^2z}=\frac{1}{x}-\frac{x}{y^2}+\frac{1}{z}-\frac{z}{y^2}$$
$$g"(y)=\frac{3y}{x^3}+\frac{3z}{x^3}>0$$
This shows that $g$ is strictly convex on $[x,z]$, and the statement follows from the fact that $g(x)=g(z)=d^2_s(a+xi,a+yi)$ and that $g(y)=d^2_s(a+xi,a+yi)+d^2_s(a+yi,a+zi)$
\end{proof}


%\textcolor{red}{Menglin, quand il y  des connecteurs autour dúne fraction, c'est mieux si tu écris ``/left/Vert,... /right/Vert'' (avec des backslashs), pareil pour les parenthèse, crochets et accolades:
%$$d_s(\Delta,\Delta')=\left\Vert \frac{(\Delta-\Delta')^2}{\Delta\Delta'}\right\Vert$$}


\begin{lemma}
    Given $\Delta,\Delta'\in \mathcal{D}_n(\mathbb{H}):$ and $\Lambda\in \mathcal{D}^+_n$
    $$d_s(\Lambda\Delta,\Lambda\Delta')=d_s(\Delta,\Delta')$$ 
    $$d_s(-\Delta^{-1},-\Delta'^{-1})=d_s(\Delta,\Delta')$$
\end{lemma}

\begin{lemma}
    Given four diagonal matrices $\Delta,\Delta',D,D'\in \mathcal{D}_n(\mathbb{H}):$
    $$d_s(\Delta+D,\Delta'+D')\leq \max(d_s(\Delta,\Delta'),d_s(D,D'))$$
\end{lemma}
\begin{proof}
    For any $\Delta,\Delta',D,D'\in \mathcal{D}_n(\mathbb{H}):$, there exist $i_0\in [n]$ such that:
    \begin{equation*}
    \begin{split}
    d_s(\Delta+D,\Delta'+D')&=\frac{|\lambda_{i_0}-\Lambda'_{i_0}+D_{i_0}-D'_{i_0}|}{\sqrt{\Im(\Delta_{i_0}+D_{i_0})\Im(\Delta'_{i_0}+D'_{i_0})}}\\
    &\leq \frac{|\lambda_{i_0}-\Lambda'_{i_0}|+|D_{i_0}-D'_{i_0}|^2}{\sqrt{\Im(\Delta_{i_0})\Im(\Delta'_{i_0})}+\sqrt{\Im(D_{i_0})\Im(D'_{i_0})}}\\
    &\leq \max\left( \frac{|\lambda_{i_0}-\Lambda'_{i_0}|}{\sqrt{\Im(\Delta_{i_0})\Im(\Delta'_{i_0})}}, \frac{|D_{i_0}-D'_{i_0}|}{\sqrt{\Im(D_{i_0})\Im(D'_{i_0})}}\right)    
    \end{split}    
    \end{equation*}
\end{proof}

In proving this property we have used the following elementary inequality results.

\begin{lemma}
    Given four positive real numbers $a,b,\alpha,\beta$:
     $$\sqrt{ab}+\sqrt{\alpha \beta}\leq \sqrt{(a+\alpha)(b+\beta)}$$
     $$\frac{a+\alpha}{b+\beta}\leq max(\frac{a}{b},\frac{\alpha}{\beta})$$
\end{lemma}
\begin{proof}
    For the first result, we deduce from the inequality $2\sqrt{ab\alpha\beta}\leq a\beta+b\alpha$:
$$(\sqrt{ab}+\sqrt{\alpha\beta})^2=ab+\alpha\beta+2\sqrt{ab\alpha\beta}\leq ab+\alpha\beta+a\beta+b\alpha$$
For the second result, we simply bound:
$$\frac{a+\alpha}{b+\beta}= \frac{a}{b}\frac{b}{b+\beta}+\frac{\alpha}{\beta}\frac{\beta}{b+\beta}\leq max\left(\frac{a}{b},\frac{\alpha}{\beta}\right).$$
\end{proof}

\begin{definition}
    Given $\lambda>0$, we denote $\mathcal{C}^{\lambda}_s(\mathcal{D}_n(\mathbb{H}))$, the class of functions $f:\mathcal{D}_n(\mathbb{H})\to \mathcal{D}_n(\mathbb{H})$, $\lambda$-Lipschitz for the semi-metric $d_s$; i.e. satisfying for all $D,D'\in \mathcal{D}_n(\mathbb{H}):$
    $$d_s(f(D),f(D'))\leq \lambda d_s(D,D').$$
    When $\lambda<1,$ we say that $f$ is contracting for the semi-metric $d_s$.
\end{definition}


\begin{proposition}
    Given three parameters $\alpha,\lambda,\theta>0$ and two mappings $f\in \mathcal{C}^{\lambda}_s$ and $g\in \mathcal{C}^{\theta}_s,$
\begin{equation*}
    \frac{-1}{f}\in \mathcal{C}^{\lambda}_s,\ \ \ \alpha f\in \mathcal{C}^{\lambda}_s,\ \ \ f\circ g\in \mathcal{C}^{\lambda\theta}_s, \ \ \ f+g \in \mathcal{C}^{\max(\lambda,\theta)}_s
\end{equation*}
    
\end{proposition}


\section{Fixed point theorem for contracting mapping}

The Banach fixed point theorem states that a contracting function on a complete space admits a unique fixed point. The extension of this result to contracting mappings on $\mathcal{D}_n(\mathbb{H})$, for the semi-metric $d_s$, is not obvious: first, because $d_s$ does not verify the triangular inequality and second because the completeness needs to be proven. The completeness is guaranteed by a boundedness condition that we impose on the matrices.

%\begin{theorem}
%    If $f:\mathcal{D}^+_n\to \mathcal{D}^+_n$ is contracting for the semi-distance $d_s$, then it's also contracting for the hyperbolic metric $d$.(\textcolor{red}{wrong without the compactness})
%\end{theorem}
%\begin{proof}
%    Let $c(r)$ be the function $c(r):=cosh^{-1}(\lambda r+1)/cosh^{-1}(r+1)$, then $c(r)<1 \ \ \forall r$ as $cosh^{-1}$ is increasing. In particular, for $r=\frac{1}{2}d_s(\Delta,\Delta')$, we obtain
%    $$cosh^{-1}(\frac{\lambda}{2}d_s(\Delta,\Delta')+1)\leq c_1 cosh^{-1}(\frac{\lambda}{2}d_s(\Delta,\Delta')+1)$$
%    Still by the increasing property we have
%    $$d_H(f(\Delta),f(\Delta'))=cosh^{-1}(\frac{1}{2}d_s(f(\Delta),f(\Delta'))+1)\leq cosh^{-1}(\frac{\lambda}{2}d_s(\Delta,\Delta')+1)\leq c_1 cosh^{-1}(\frac{\lambda}{2}d_s(\Delta,\Delta')+1)$$
%\end{proof}

\begin{theorem}
Given a subset $\mathcal{D}_b$ of $\mathcal{D}_n(\mathbb{H})$ where each diagonal entry has an imaginary part bounded from above and below and a mapping $f:\mathcal{D}_b \to \mathcal{D}_b$ , if it is furthermore contracting for the stable semi-metric $d_s$ on $\mathcal{D}_b$, then there exists a unique fixed point $\Delta^*\in \mathcal{D}_b$ satisfying $\Delta^*=f(\Delta^*)$. 
\end{theorem}
\begin{proof}
Noting $\lambda \in(0,1)$ the Lipschitz constant such that $\forall \Delta,\Delta'\in \mathcal{D}_n(\mathbb{H}), d_s(f(\Delta),f(\Delta'))\leq \lambda d_s(\Delta,\Delta'),$ we show that the sequence $(\Delta^{(k)})_{k\geq 0}$ satisfying:
$$\Delta^{(0)}=I_n, \ \ \ \forall k\geq 1, \Delta^{(k)}=f(\Delta^{(k-1)})$$
is a Cauchy sequence in $\Bar{\mathcal{D}}_n(\mathbb{H})$, where $\Bar{\mathcal{D}}_n(\mathbb{H})\equiv \mathcal{D}_n(\mathbb{H}\bigcup \mathbb{R}).$
$\forall p\in\mathbb{N}$, $\Delta^{(p)}\in \mathcal{D}_b$, i.e. there exists $\delta>0$, such that $|\Im \Delta^{(p)}|\leq \delta.$
We can then bound for any $p\in\mathbb{N}:$
$$\Vert \Delta^{(p+1)}-\Delta^{(p)}\Vert \leq \delta d_s(\Delta^{(p+1)},\Delta^{(p)})\leq \lambda^p \delta d_s(\Delta^{(1)},\Delta^{(0)}).$$
Therefore, thanks to the triangular inequality in $(\mathcal{D}_n(\mathbb{H}),\Vert\cdot\Vert),$ for any $n\in\mathbb{N}:$
\begin{equation*}
\begin{split}
\Vert \Delta^{(p+n)}-\Delta^{(p)}\Vert &\leq \Vert \Delta^{(p+n)}-\Delta^{(p+n-1)}\Vert+\cdots +\Vert\Delta^{(p+1)}-\Delta^{(p)}\Vert \\
&\leq \frac{\delta d_s(\Delta^{(1)},\Delta^{(0))}}{1-\lambda}\lambda^p\to 0.    
\end{split}
\end{equation*}
This allows us to conclude that $(\Delta^{(p)})_{p\in \mathbb{N}}$ is a Cauchy sequence, and therefore it converges to a diagonal matrix $\Delta^{*}\equiv \lim_{p\to\infty} \Delta^{(p)}\in \Bar{\mathcal{D}}_n(\mathbb{H})$ which is a closed thus complete set. But since $\Delta^{(p)}$ has diagonal entries which are bounded from below, we know that $\Delta^*\in \mathcal{D}_b$.
By contractivity of $f$, it is clearly unique.
\end{proof}

\section{Stability of the stable semi-metric towards perturbations}
We have first of all the following elementary inequality result.
\begin{lemma}
    Given theree diagonal matrices $\Gamma^1,\Gamma^2,\Gamma^3\in\mathcal{D}_n(\mathbb{H})$:
    $$\left\Vert \frac{\Gamma^3}{\sqrt{\Im(\Gamma^1)}} \right\Vert\leq \left\Vert \frac{\Gamma^3}{\sqrt{\Im(\Gamma^2)}}(1+d_s(\Im(\Gamma^1),\Im(\Gamma^2))) \right\Vert$$
\end{lemma}
\begin{proof}
    We simply bound:
     \begin{equation*}
    \begin{split}
    \left\Vert \frac{\Gamma^3}{\sqrt{\Im(\Gamma^1)}} \right\Vert 
    &\leq  \left\Vert \frac{\Gamma^3}{\sqrt{\Im(\Gamma^2)}} \right\Vert 
+\left\Vert \frac{\Gamma^3 \left(\sqrt{\Im(\Gamma^2)}-\sqrt{\Im(\Gamma^1)}\right)}{\sqrt{\Im(\Gamma^2)\Im(\Gamma^1)}} \right\Vert \\
    &\leq \left\Vert \frac{\Gamma^3}{\sqrt{\Im(\Gamma^2)}} \right\Vert 
+\left\Vert \frac{\Gamma^3}{\sqrt{\Im(\Gamma^2)}} \right\Vert \left\Vert \frac{\Im(\Gamma^2)-\Im(\Gamma^1)}{\sqrt{\Im(\Gamma^1)}\left(\sqrt{\Im(\Gamma^2)}+\sqrt{\Im(\Gamma^1)}\right)} \right\Vert
    \end{split}
    \end{equation*}
\end{proof}

Next we give the result to bound the distance between a diagonal matrix and the other one which is obtained as a fixed point.
\begin{proposition}
    Given a diagonal matrix $\Gamma\in \mathcal{D}_n(\mathbb{H})$, a mapping $f:\mathcal{D}_n(\mathbb{H})\to \mathcal{D}_n(\mathbb{H})$ $\lambda$ contractive for the semi-metric $d_s$ with the Lipschitz coefficient $\lambda<1$ and admitting the fixed point $\Tilde{\Gamma}=f(\Tilde{\Gamma})$, we have the bound:
    \begin{equation*}
        d_s(\Gamma,\Tilde{\Gamma})\leq \left\Vert \frac{f(\Gamma)-\Gamma}{\sqrt{\Im(\Gamma)\Im(\Tilde{\Gamma})}}\right\Vert/(1-\lambda-\lambda d_s(\Im(\Gamma)),\Im(f(\Gamma))) 
    \end{equation*}
\end{proposition}
\begin{proof}
    Thanks to the above lemma, we can bound:
    \begin{equation*}
    \begin{split}
        d_s(\Gamma,\Tilde{\Gamma})&\leq \left\Vert \frac{\tilde{\Gamma}-f(\Gamma)}{\sqrt{\Im(\Gamma)\Im(\Tilde{\Gamma})}}\right\Vert+\left\Vert \frac{f(\Gamma)-\Gamma}{\sqrt{\Im(\tilde{\Gamma})\Im(\Gamma)}} \right\Vert\\
        &\leq d_s(\tilde{\Gamma},\Gamma)(1+d_s(\Im(\Gamma)),\Im(f(\Gamma))+\left\Vert \frac{f(\Gamma)-\Gamma}{\sqrt{\Im(\tilde{\Gamma})\Im(\Gamma)}} \right\Vert\\
        &\leq \left\Vert \frac{f(\Gamma)-\Gamma}{\sqrt{\Im(\Gamma)\Im(\Tilde{\Gamma})}}\right\Vert/(1-\lambda-\lambda d_s(\Im(\Gamma)),\Im(f(\Gamma))) 
    \end{split}
    \end{equation*}
\end{proof}
\begin{proposition}
    Let us consider a family of mappings$(f^m)_{m\in\mathbb{N}}$ of $\mathcal{D}_{n_m}(\mathbb{H})$, each $f^m$ being $\lambda$-Lipschitz for the semi-metric $d_s$ with $\lambda<1$ and admitting the fixed point $\Tilde{\Gamma}^m=f^m(\Tilde{\Gamma}^m)$ and a family of diagonal matrices $\Gamma^m$. If one assume that $d_s(\Im(\Gamma^m),\Im(f^m(\Gamma^m)))\leq o_{m\to \infty}(1)$, then 
    $$d_s(\Gamma^m,\Tilde{\Gamma}^m)\leq O_{m\to\infty}\left(\left\Vert\frac{f^m(\Gamma^m))-\Gamma^m}{\sqrt{\Im(\Tilde{\Gamma^m})\Im(\Gamma^m)}}\right\Vert\right)$$
\end{proposition}
\begin{proof}
    For $m$ sufficiently big, we have $d_s(\Im(\Gamma^m),\Im(f^m(\Gamma^m)))\leq o(1)\leq \frac{1-\lambda}{2\lambda}$, so we have:
    \begin{equation*}
    \begin{split}
d_s(\Gamma^m,\Tilde{\Gamma}^m)& \leq \left\Vert \frac{f(\Gamma^m)-\Gamma^m}{\sqrt{\Im(\Gamma^m)\Im(\Tilde{\Gamma}^m)}}\right\Vert/(1-\lambda-\lambda d_s(\Im(\Gamma^m)),\Im(f(\Gamma^m))) \\
&\leq \left(\left\Vert\frac{f^m(\Gamma^m))-\Gamma^m}{\sqrt{\Im(\Tilde{\Gamma^m})\Im(\Gamma^m)}}\right\Vert\right)
    \end{split}
    \end{equation*}
\end{proof}

\end{document}
\end{document}
